\documentclass{llncs}
\usepackage{makeidx}
\usepackage[pdftex]{hyperref}
\begin{document}
\frontmatter
\pagestyle{headings}
\title{Using MDD to specify a DSL for a Policy Engine grounded in Facility Management requirements}
\titlerunning{Building Policy Engine}
\author{Hansen, K.\, Kontostathis, K., Schmidt, E.}
\authorrunning{Hansen, K et al.}
\tocauthor{Hansen, K., Kontostathis, K., Schmidt, E.}
\institute{IT-University of Copenhagen, Denmark,\\
\email{kben@itu.dk}, \email{kkon@itu.dk}, \email{eker@itu.dk}}
\maketitle
\section{Project Proposal}
Natural resources are a precious commodity and are to be used sustainably. Constructing resource efficient buildings makes sense, both in a political and economical perspective. Modern buildings today might come equipped with a suite of sensors and actuators, opening up for a degree of customizable control. Our collective need is that buildings can adapt to the users and the sensor-perceived environment, either automatically or manually. This can be achieved by developing policies based on input like semi-static data, dynamic data and sensor input, which control the actuators thereby leading to the desired building state.

Since the task of specifying building policies typically resides in the area of Facility Management, we find it pertinent to interview company employees from positions within. We will conduct open-ended interviews and document these by audio and/or video. The main purpose of the interviews is to document existing building policies and facilitate the discovery of new, relevant ones without taking the limitations of technology (like the expressiveness of a programming language) into account. We will analyse the interviews, and use both Eclipse and EMF to build a meta model for a domain specific language that is rich enough to express the policies. Using the model and XText, we will develop the grammar needed, in order to have syntactical code completion in a running a Policy Engine Editor based on Eclipse. Finally, we will use the Policy Engine Editor to write the policies.

We therefore base the validity of this project on the requirements defined by the professionals working in Facility Management, and use their stated policies as requirements to be implemented as grammar-passed code using the developed DSL.

If time permits, we might develop parts of the execution, compare the DSL to existing available software solutions and develop a visual editor.
\end{document}