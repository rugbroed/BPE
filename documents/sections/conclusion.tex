In our point of view, the proposed DSL constitutes a useful tool to every simple user planning to define automation policies for his residence or working place. The supporting arguments are the user-friendly interface and the self-explanatory keywords alongside with the flexibility of the abstraction that the DSL provides. Naturally, its scope is narrowed and there is no actual implementation yet, mainly due to the size of the team and the limitations of the semester, but nevertheless it can potentially be improved in many ways as described in ~\nameref{sec:discussion} and lead to the development of an energy-efficient ally of our welfare in both a personal and global dimension.
