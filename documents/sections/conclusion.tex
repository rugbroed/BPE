Even though the domain of building automation is large, and some existing work is using MDD concepts, no high-level DSL abstraction exists. Our DSL, that is the main contribution to the domain of building automation, addresses this gap.

We have conducted six well-prepared interviews with different danish companies to examine some of their needs with regards to building automation. We focused on \textit{policies} --- building automation governing pieces of functionality, specifically what the interviewees already are using and what they would like to see implemented if nothing were impractical or infeasible. Abstract and concrete syntax was iteratively designed and implemented, using an analysis of the interviews as a foundation for requirements. Both were equally refined to have a simplistic, conceivable and professional look. 

Our DSL is designed to be a useful tool to anyone residing in facility management or doing building automation. It was designed using already familiar concepts as keywords, such as \textit{policies}, \textit{room}, \textit{floor}, placing the language at a higher abstraction level than existing technologies --- and thereby introducing novelty in the domain. The language contains a good amount of flexibility, both in regards to the structure of the automation, the temporal aspect and the possibility to define \textit{room types}. We argue that this increase adaptability and manageability. Great care has been taken to implement the meta-model in such a way that good auto-completion is offered the user, leading to fast, easy and infallible written policies. 

The DSL was evaluated based on what was achievable using the DSL in regards to core functionalities extrapolated during the analysis of the interviews. We show that the important core functionalities are achievable using our DSL and argue that our DSL constitutes a novel perspective on building automation. With further improvement, our policy engine has the potential to significantly contribute to the domain of building management.