Summarizing briefly, we initially conducted well-prepared interviews with some danish companies to find out their needs for building automation systems --- what they already have in use, how they can be improved and what they needed to further improve. Then the meta-model was created and subsequently the abstract and concrete syntax. Later on, we refined both grammars interchangeably to fit our interviewees needs and have a more simplistic, conceivable and professional look. Evaluation of our DSL relied on how close we got in our effort to reach the intended goals. Fortunately, all the building specifications, policies definitions can be defined and automated. Everything else constitutes  \nameref{sec:futureWork} (section\ref{sec:futureWork}).

Our DSL is designed to be a useful tool to any simple user planning to define automation policies for his residence or working place. The supporting arguments are the user-friendly interface and the self-explanatory keywords alongside with the flexibility of the abstraction that the DSL provides. Furthermore, the user can define his own components, rooms and policies, which highly increase adaptability and portability of our language. Last but not least, auto-completion leads to fast, easy and infallible written policies. With further improvement, our policy engine has the potential to significantly contribute in our welfare in both a personal and global perspective.
