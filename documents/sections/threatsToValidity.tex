\subsection{Representativeness}

The main threat to the external validity of our developed DSL is that the interviews were restricted only to four danish companies. This was unavoidable due to our time pressure and the heavy workload of the companies employees. As a result we should be careful in generalizing our language for every company, especially for buildings that are located in distant lands with different climatological and cultural conditions. On the other hand, all four interviewed companies are largely independent, specialized on different fields and in no way correlated, therefore the sample is quite objective and reliable. Furthermore, it is always possible for the user to define his own sensors and actuators in regards to his needs.

\subsection{Assessment of our own work}

Another external risk of our project is that we are the judges of our own work, which makes our evaluation appear more inclined to our favor. Our initial plan was to ask the same participants and/or some fellow students evaluate our DSL, but this required an abundance of time which neither we nor the interviewees were able to provide. As a consequence, we decided to formulate the evaluation as a comparison of what it should have been implemented -what they asked for in the beginning- and what it was actually developed in the end, while trying to maintain an unbiased perspective at the highest level possible.

\subsection{Insignificant Statistics}

Last but not least, a minor internal threat related to the representativeness of our DSL is the limited number of the interviews. Had we conducted more interviews on diverse companies, we might have a better understanding of their needs, for instance which components are most necessary or preferred. Still, the user's ability to define new components, rooms, policies etc. makes our language so flexible that eliminates the two aforementioned validity risks successfully.