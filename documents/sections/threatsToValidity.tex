\subsection{Representativeness}
The main threat to the validity of our developed DSL is that the interviews were restricted only to the six danish companies as mentioned in \nameref{sec:interviewsAndAnalysis} (section \ref{sec:interviewsAndAnalysis}). We do not know if the educed requirements would be representative enough for a larger base of customers as the derived work is for these particular companies. On the other hand we can argue that based on the organizations and buildings from the interviews, they offer failry diverse commercial building(s) that we believe cover a fairly broad ground to base our project on.

\subsection{Assessment of our own work}
Another risk to the validity of our project is that we are the judges of our own work, which makes our evaluation appear more inclined to our favor. In a more proper setup it would have been ideal to ask the interviewees to evaluate it but due to time constraints, this was infeasible. As a consequence, we decided to formulate the evaluation as a comparison of what was possible to achieve using our DSL to implement the elicited requirements and what the requirements were while trying to maintain an unbiased perspective at the highest level possible. 

%DO NOT SEE THE RELEVANCE OF THIS PART BELOW------------------------------
\subsection{Insignificant Statistics}

Last but not least, a minor internal threat related to the representativeness of our DSL is the limited number of the interviews. Had we conducted more interviews on diverse companies, we might have a better understanding of their needs, for instance which components are most necessary or preferred. Still, the user's ability to define new components, rooms, policies etc. makes our language so flexible that eliminates the two aforementioned validity risks successfully.