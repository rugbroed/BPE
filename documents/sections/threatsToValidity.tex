\subsection{Representativeness}
The main threat to the validity of our developed DSL is that the interviews were restricted only to the six danish companies as mentioned in \nameref{sec:interviewsAndAnalysis} (section \ref{sec:interviewsAndAnalysis}). We do not know if the educed requirements would be representative enough for a larger base of customers as the derived work is for these particular companies. On the other hand we argue that --- based on the organizations, buildings and policies described in the interviews --- our DSL most probably would be rich and expressive enough to satisfy a large group of companies.

\subsection{Assessment of our own work}
Another risk to the validity of our project is that the evaluation is done by ourselves --- though based on actual requirements from the interviews --- which makes the evaluation biased. In a more proper setup it would have been ideal to ask the interviewees to evaluate the DSL but due to time constraints (both on our part and the companies involved), this was infeasible. As a consequence, we decided to tailor the evaluation on core requirements (requirements --- elicited by us --- that seemed advanced and were reoccurring amongst interviewees) and implement those using our DSL.