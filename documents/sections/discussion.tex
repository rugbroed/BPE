During the design and implementation of our metamodel we made many different design decisions, primarily based on the length and scope of this project and revelations from the interviews. Some of these decisions merits explanations.

\subsection{Expression language}
Since the expression language needed for our DSL is pretty simple, we chose to implement it ourselves. We also looked at basing our expression language on the functionality provided by XBase \cite{xbase}, however the current state of XBase seems far from a stable product. Since our timeframe was rather limited, we did not want to use our time with the implementation details of XBase. We expect XBase to further develop as time goes, and later it might be an excellent choice in regards to the expression language.

\subsection{Class use versus instance use}
As can be observed in Figure \ref{fig:dsl-policy-definition} we rely heavily on class use instead of instance use. For example the sentence "uses sensors TemperatureSensor" uses a TemperatureSensor as a class, and not as an instance. We do not find this anymore uncommon than using, for example, static functions in java classes. It is simply a way to achieve lots of fast and easy functionality in a non-complicated way. Imagine that we had to define a 10 story building, room by room, sensor by sensor, actuator by actuator and so forth. The source would quickly become littered with instances, which probably would need to carry the name of the room in it's name in order for the user to keep track of them. An example could be the fictive sensor \textit{buildingFactory3rdFloorRoomNextToKitchenTemperatureSensor}. By using classes directly, we can define room types constructed from both their common sensors and actuators, but also on their need for governing policies. It is still possible to use instances, as also shown in Figure \ref{fig:dsl-policy-definition} (\textit{janitorOffice}) if we, for example, need to activate \textit{a single} actuator based on input from several room sensors. Aspects of this should be adressed in future work (see Section \ref{subsec:collections}).
