A lot of research has already been conducted in home and building automation, spanning from low level communication protocols like BACnet, LonWorks and EIB/KNX~\cite{communication} to full system implementations. In comparison, less progress has been made on user friendly DSL designs, where everyday building concepts are mixed with time constraints and conditional logic for building automation. 

In~\cite{smartscript}, a DSL named SmartScript is developed. Based on a publish/subscribe paradigm it is intended for appliance control, and implemented with a KNX adapter. It allows grouping of devices to make it possible for instances to turn on the light and dim it using the same `light' concept, even though the light switch and dimmer are two separate electronic components. The language features three types of statements; Action, If and Loop. Specifically, there are two actions; Set and Get. Although it is possible to use the group concept, it does not seem to be possible to define the entire building with rooms based on different types. This results in a long collection of variables, even for small and simple buildings. Moreover, the script seems to be operating on a much lower abstraction level than our DSL, maybe due to its implementation that seems to have pulled the DSL more into the direction of a GPL\footnote{General Purpose Language}. 

In~\cite{habitation}, there is a clear distinction between usage of the domain experts (called the Catalog View) and the domain users (called the Application view). A graphical editor is used when designing an application, by dragging and dropping types into flows that will be orderly executed. However, the user designing the application still has to get familiar with concepts such as \textit{standard functional units}, \textit{FUnitLinks} and \textit{Scenes}. Again, our DSL is at a higher abstraction level. However, the catalog view might be considered as the current state of our metamodel, where the different concepts are defined. Our model only operates on concepts already known to people expected to work with building automation like \textit{Building, Floor, Room, Sensor, Actuator, Schedule, and Policy}. Apart from the obvious differences, our DSL is text-based, versus the graphical editor in Habitation, our flow resides in the combinatory logic derived from time schedules, room types, boolean states and conditional logic specified in the policies. This will become evident in \nameref{sec:designAndDevelopment} (section~\ref{sec:designAndDevelopment}).

In the Google sponsored Home Automation Bus --- openHab~\cite{openhab} it is also possible to specify rules, running on actual implemented hardware. OpenHAB offers a whole suite of implemented protocol standards and functionalities, that includes an Xtext based script interpreter. OpenHAB's rule language is, like most other systems, very low level and does not offer the expressive benefits of a high abstraction DSL.