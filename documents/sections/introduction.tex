Energy and natural resources are precious commodities. In~\cite{janssen2004towards} it is stated that residential buildings use about 89\% of the total energy consumption for space heating, cooling and water heating. Electrical appliances use 11\%. Other buildings use 79\% of the total energy consumption for space heating, cooling, water heating and lighting. To \textit{manually} control buildings for energy efficiency and optimal human comfort is a time-consuming and inefficient task, if even possible. The controller needs to possess a diverse knowledge of the necessary equipment and the building's characteristics at the same time. Modern buildings today often come equipped with a suite of sensors and actuators, opening up for a degree of customizable control. Building automation is therefore not only possible, but also necessary. \textit{``Worldwide, there is no doubt that efficient energy saving is only possible with modern BA based on networking in all levels of abstraction."}~\cite{dietrich2010communication}. Constructing resource efficient buildings makes sense, both in a political and economical perspective. 

Consequently, our collective need is the adaptation of buildings to the users and the sensor-perceived environment. This can be achieved by developing governing \textit{policies} (defined as pieces of code) based on input such as semi-static data, dynamic data and sensor input, which control the actuators and thereby leading to the desired building state.

Merriam-Webster defines \textit{policy} as ``a definite course or method of action selected from among alternatives and in light of given conditions to guide and determine present and future decisions."

For this paper, many companies were interviewed regarding their need for building automation \textit{policies}. After analyzing the interviews and iteratively specifying a DSL for building automation, we argue by example, that our proposed DSL can actually satisfy the needs and the intended requests made during the interviews.

The structure of the paper is as follows. We start by discussing all \nameref{sec:relatedwork} and compare it with our own. We then move on to~\nameref{sec:method}, followed by a section on~\nameref{sec:dsldesign}. Later on we explain our~ \nameref{sec:evaluation}, and its results in the~\nameref{sec:discussion} section. Last comes the summing up of our findings in~\nameref{sec:conclusion}.