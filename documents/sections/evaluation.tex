The requirements received from Br\"{u}el \& Kj\ae r were implemented in our DSL and can be found in \nameref{sec:dsl-bruel}. One of the requirements from the interview (\textit{all sensor and actuator data stored in a repository for use in a visual display of data in the FM office}) was not implemented in our DSL because we decided to narrow the scope of the project. After evaluating these requirements, we realized that some of the sensors and actuators needed to satisfy the implementation are not defined in our metamodel. While these problems could be easily fixed by simply adding some new classes in the metamodel, we evaluated this area as one that our DSL can be further improved as discussed in \nameref{subsec:def-sensor-actuator-types}.

The implementation of the requirements from K\o benhavns Tekniske Skole can be found in \nameref{sec:dsl-kts} and just like in the evaluation in \nameref{subsec:bruel}, we faced the same lack of sensors and actuators. This seems to be a recurring issue in our language but nonetheless it provides a nice opportunity to improve our language. It is further discussed in \nameref{subsec:def-sensor-actuator-types}. 
Another aspect of our language that can be refined is the ability to define how each policy should be used as discussed in \nameref{subsec:during}. 
