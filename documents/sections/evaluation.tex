The ideal methodology would have been to have these organizations evaluate our DSL but due to time constraints on our project, this was not possible and the evaluation of our DSL was mainly based on the modelling of buildings and policies identified in the interviews. The evaluation is mainly focused on 2 organizations; Br\"{u}el \& Kj\ae r and K\o benhavns Tekniske Skole. The reason for this being the size of the organization, diversity of building usage and that they offered the most advanced behaviour put forth during the interview. 

Our DSL can capture the most important parts of the
organization's buildings and can define most of the policies identified. By using this methodoly, we were able to evaluate our dsl in 2 iterations; The first outcome of the evaluation offered both positive and negative outcome which are described in more details below.\\

In Br\"{u}el \& Kj\ae r, our DSL can capture all parts of the organization's buildings and most of the policies identified in the interview. One of the requirements was not implemented because we had to narrow the scope of the project due to time constraints. (\textit{"This involved storing all the sensor data in a database and to be further used by a web-based system to visualize the data in the janitors office and mobile devices"}).\\
In \nameref{app:bogk}, our DSL manages to captures the building specification as it is; 2 buildings, each with 2 floors and each floor with some rooms. we were able to predefine rooms of certain types and extending some of the rooms defined in the building model to extend these types. The DSL also captures most of the policies identified; Schedules and policies are defined. These policies make use of the actuator and sensor types, makes use of the predefined rooms and in the implementation of the section, we use the expression language to check sensor values and based on that we set actuator types or specific instances in specific rooms to a desired behaviour. \\

In  \nameref{app:kts}, K\o benhavns Tekniske Skole, just as in the case above, our DSL captured most of the policies and predefined room types as in the interview. In this evaluation we focus more on modelling policies, use of states and timers and room-types.\\ 

After evaluating policies and building specification of both organizations, we realized that some of the sensors and actuators needed to satisfy the implementation were not defined in our metamodel. While these problems could be easily fixed by simply adding some new classes in the metamodel, we evaluated this area as one that our DSL can be further improved as discussed in Section \ref{subsec:def-sensor-actuator-types}. Another negative discovered in the evaluation of K\o benhavns Tekniske Skole is the posibility to specify a schedule when each policy should be run individually as discussed in Section \ref{subsec:during}. The final prolem in our DSL was our expression language. We needed to implement negation of expressions, better handle recursive expressions and implement the use of statements in our DSL.\\

The positives of the first iteration being, the building specification was spot on and the buildings could be modelled precisely and policies could be partly specified. We were also able to find weaknesses, which could be fixed in time and areas that can be further developed in Section \ref{sec:futureWork}. The second evaluation, saw the same policies modelled and the outcome being an error free and accurate implementation of the building and policies using our DSL. 