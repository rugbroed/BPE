The evaluation of our DSL was conducted iteratively and based on the successful modeling --- using the DSL --- of buildings and policies identified in the analysis of the interviews with the two organizations; Br\"{u}el \& Kj\ae r and K\o benhavns Tekniske Skole, mainly because they contain the most advanced requirements for BA. Another method of evaluation would have been to have these organizations evaluate our DSL, but due to time constraints on both our project and in the companies this was not possible. 

Our DSL can capture the most important parts of the organization's building specification and can define the policies identified (not involving existing custom subsystems as described in ~\ref{sec:introduction}). 

In Br\"{u}el \& Kj\ae r, our DSL can capture all parts of the organization's buildings and all the policies identified in the interview. One of the requirements was not implemented due to it being outside the scope --- \textit{store all the sensor data in a database in order to be further used by a web-based system to visualize the data in the janitors office}.

In \nameref{app:bogk}, our DSL manages to capture the building specification as it is; two buildings, each consisting of two floors and each floor of several rooms. We were able to define rooms types and extend rooms to be of those types. The DSL also captures most of the policies identified; \textit{schedules} and \textit{policies} are defined. The policies make use of both actuators and sensors as well as the room types. In the implementation we use the expression language to check sensor values and based on that we set actuator types or the defined instances in specific rooms to a desired state.

In \nameref{app:kts}, K\o benhavns Tekniske Skole, just as in the case above, our DSL captured most of the policies and predefined room types as in the interview. In this evaluation we focus more on modeling policies, use of states, timers and room-types.

During an evaluation iteration we we realized that some of the sensors and actuators needed to satisfy the implementation were not defined in our meta-model. Prioritizing our time, we chose to focus our attention elsewhere --- and this could have been easily fixed by simply adding the new classes in the meta-model, as discussed in \nameref{subsec:def-sensor-actuator-types} (section \ref{subsec:def-sensor-actuator-types}). A missing feature was discovered in a later evaluation iteration of the policies presented in K\o benhavns Tekniske Skole. It would have been optimal to chain together a given policy and it's schedule, instead as the current \textit{during concept} where a bunch of policies is sharing a schedule. This is discussed further in \nameref{subsec:during} (section \ref{subsec:during}). The expression language was also found to be lacking important functionality in the earlier iterations --- for example to handle recursive expressions and the negation of expressions --- but was later improved and now works as depicted in the DSL examples. 

Using our iterative method of evaluation against defined policies from interviews enabled us to find weaknesses, which could be fixed in time, as well as areas that can be further developed in \nameref{sec:futureWork}. (section \ref{sec:futureWork}). 