The ideal methodology would have been to have these organizations evaluate our DSL but due to time constraints on our project, this was not possible and the evaluation of our DSL was mainly based on the modeling of buildings and policies identified in the interviews. The evaluation is mainly focused on 2 organizations; Br\"{u}el \& Kj\ae r and K\o benhavns Tekniske Skole. The reason for this is the size of the organization, the diversity of building usage and our interest in their requirements to be modeled. 

Our DSL can capture the most important parts of the organization's buildings and can define most of the policies identified. By using this methodology, we were able to evaluate our DSL in 2 iterations; The first outcome of the evaluation offered both positive and negative results which are described in more details below.\\

In Br\"{u}el \& Kj\ae r, our DSL can capture all parts of the organization's buildings and most of the policies identified in the interview. One of the requirements was not implemented because of our narrowing the scope of the project due to time constraints. (\textit{``This involved storing all the sensor data in a database in order to be further used by a web-based system to visualize the data in the janitors office and mobile devices"}).\\
In \nameref{app:bogk}, our DSL manages to capture the building specification as it is; 2 buildings, each consisting of 2 floors and each floor of several rooms. we were able to predefine rooms of certain types and extend some of them defined in the building model to extend these types. The DSL also captures most of the policies identified; Schedules and policies are defined. These policies make use of the actuator and sensor types as well as the predefined rooms. In the implementation of the section, we use the expression language to check sensor values and based on that we set actuator types or defined instances in specific rooms to a desired behaviour. \\

In \nameref{app:kts}, K\o benhavns Tekniske Skole, just as in the case above, our DSL captured most of the policies and predefined room types as in the interview. In this evaluation we focus more on modeling policies, use of states, timers and room-types.\\ 

After evaluating policies and building specification of both organizations, we realized that some of the sensors and actuators needed to satisfy the implementation were not defined in our metamodel. While these problems could be easily fixed by simply adding some new classes in the metamodel, we evaluated this area as one that our DSL can be further improved as discussed in \nameref{subsec:def-sensor-actuator-types} (section \ref{subsec:def-sensor-actuator-types}). Another minus discovered in the evaluation of K\o benhavns Tekniske Skole is the possibility to specify a schedule when each policy should be run individually as discussed in \nameref{subsec:during} (section \ref{subsec:during}). The final problem in our DSL was our expression language. We needed to implement negation of expressions, better handle recursive expressions and implement the use of statements in our DSL.\\

The positives of the first iteration being, the building specification was spot on, the buildings could be modeled precisely and policies could be partly specified. We were also able to find weaknesses, which could be fixed in time and areas that can be further developed in \nameref{sec:futureWork}. (section \ref{sec:futureWork}). The second evaluation, saw the same policies modeled and the outcome being an error free and accurate implementation of the building and policies using our DSL. 