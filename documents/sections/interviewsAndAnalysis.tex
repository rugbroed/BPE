Since building control resides with FM we found it pertinent to interview company employees working in or close to that field. We conducted open-ended interviews with these companies and documented their content mostly by handwritten notes or in a single case by audio. The main purpose of the interviews was to document;

\begin{enumerate}
	\item existing governing rules, implicit or explicit.
	\item existing and requested, sensors and actuators.
	\item requests for new governing rules regardless of their practicality or feasibility.
\end{enumerate}

In order to gather enough material to conceptualize a proper DSL, we interviewed people from the following companies;

\subsubsection{Br\"{u}el \& Kj\ae r} supplies integrated solutions for the measurement and analysis of sound and vibration with facilities located in N\ae rum, Denmark. These facilities total up to an area of approximately 20,000$m^2$ and are divided into 2 building with 2 floors each; 
\begin{itemize}
	\item Main building is used for office space, calibration of new products, research and development,``university" which is used for seminars and conferences, cafeteria and storage space.
	\item Repair building is only used for repairs and testing of products.
\end{itemize}	
These facilities are daily used by 500 employees for various functionality and only 5 are employed in the FM Department to maintain the buildings. Some of the policies in place are; \texttt{LightControl}---in all toilets,  \texttt{TemperatureControl}---in offices, production, calibration, repair and \texttt{HumidityContol}---in production, calibration and testing rooms.
\begin{description}
	\item[Interviewee] Building Maintanance Manager.
	\item[Duration] 1 hour in total, face to face
	\item[Interview] Audio Recording
\end{description}

\subsubsection{Bygningsstyrelsen} has the responsibility for the management of some of Denmark's important state institutions (for example \textit{universities}, \textit{ministries} and \textit{governmental offices}) and for parts of the legislation in that domain. The interview started by asking about the least and the most advanced buildings they manage;
\begin{itemize}
	\item Even the least advanced buildings feature existing CTS systems, and are primarily used for heating only. No other sensors or actuators might be available.
	\item The most advanced buildings features many different types of sensors and actuators, still including a CTS system for temperature control and sometimes for other purposes as well. 
\end{itemize}	
\begin{description}
	\item[Interviewee] Chief Consultant (Chefkonsulent), Manager
	\item[Duration] 2 hours in total by phone
	\item[Interview] Transcribed
\end{description}

\subsubsection{IT-University of Copenhagen} is an educational institution based in Copenhagen. The facility is one main campus with an approximate area of 19,000$m^2$. The building has a basement and 5 floors.
\begin{itemize}
	\item The basement is used for bicycle parking and gym facilities.
	\item The ground consists of an atrium, canteen, lecture rooms and offices.
	\item Like the ground floor, the rest of the floors consist of lecture rooms and offices.
\end{itemize}
\begin{description}
	\item[Interviewee] Building Maintenance Manager.
	\item[Duration] 1 hour
	\item[Interview] Transcribed
\end{description}

\subsubsection{K\o benhavns Tekniske Skole} is an educational institution based in Copenhagen with colleges in various parts of the city. The interview was conducted with an employee in the school at Drag\o r, approx. 9km outside Copenhagen.
\begin{itemize}
	\item The basement is used for storage of furniture and disused equipment.
	\item The ground floor features classrooms, rooms for teachers, kitchens, toilets, hallways, a canteen, and rooms for various technical equipment like computers, photographical equipment etc.
	\item The top floor contains a hallway with skylights and more classrooms and offices.
\end{itemize}
\begin{description}
	\item[Interviewee] V\ae rkstedsassistent (Workshop assistant).
	\item[Duration] 1,5 hours in total by phone
	\item[Interview] Transcribed
\end{description}

\subsubsection{ST Aerospace - Denmark} is an independent global company, offering third party aviation maintenance, repair and overhaul. They have facilities in Copenhagen, near the airport, and the building has 5 floors. Details about their building were not discussed, since the management of the building is provided by a third party --- however general concepts were extrapolated from the interview.
\begin{description}
	\item[Interviewee] Head of FM.
	\item[Duration] 0,5 hour by phone
	\item[Interview] Transcribed
\end{description}

\subsubsection{UNI-C} is an agency under The Danish Ministry of Children and Education. They promote digital development within the area of children and education. Their primary focus is to increase the use of IT in education, and supporting an effective operation of institutions by using IT. UNI-C has just moved into a refurbished and modern building with 5 floors. It is located close to the central town square (R\aa dhuspladsen) featuring ventilation system, and automated blinds.
\begin{itemize}
	\item All the floors, except the roof terrace, contains offices.
	\item The basement is used for storage.
	\item 4th floor has a big canteen.
	\item Some of the floors have conferencing rooms.
\end{itemize}
\begin{description}
	\item[Interviewee] Kontorfuldmægtig (administrative officer)
	\item[Duration] 1 hour
	\item[Interview] Transcribed
\end{description}

\subsubsection{Questions} were formulated by the team to be used as a base for these interviews and aimed at gathering relevant information for our project. Some of the questions used in the conducted interviews are shown below:
\begin{enumerate}
	\item How are buildings divided and what is the total floor area?
	\item What is the facility used for? 
	\item How many employees make use of the facility?
	\item How many work in FM department?
	\item What tools, if any, do you make use of currently?
	\item What policies are in place today?
	\item Which of them would you like to have implemented?
	\item What sensors and actuators are in use today and which others do you find relevant and necessary to have?
	\item What other implementations would you consider to make the management of the facilities easier?
\end{enumerate}

While conducting the interviews, we pointed out that practicality or feasibility should not influence the requests for new governing rules. This proved hard for the interviewees. Some interviews were therefore conducted over several sessions, giving the interviewees time to think creatively. We only used natural languages (English and Danish) to give examples of policies to facilitate the discovery of new, relevant ones. This was done to avoid the limitations of technology --- like the expressiveness of a programming language and to make communication less rigid.\\ 

Finally, this project is solely based on the conceptual requirements extrapolated from the interviews. We use the state policies as the definition on how to design the DSL. The successful grammar parsing therefore constitutes a the working DSL. For example, Br\"{u}el \& Kj\ae r requested that in case a humidity sensor in the production area drops below a certain value, the alarm in the janitors office should go off. This is implemented and can be seen in fig.\ref{fig:dsl-policy-definition}. In \nameref{sec:designAndDevelopment} (section\ref{sec:designAndDevelopment}), we elaborate more on how these interviews contributed to the design of the DSL.